\documentclass[12pt]{article}
\usepackage[a4paper,margin=1in]{geometry}
\usepackage{fancyhdr}
\usepackage{setspace}
\usepackage{titlesec}
\usepackage{fontspec}
\usepackage{xcolor}
\usepackage{marginnote}

% Font: Government-used font (Times New Roman or Nimbus Roman)
\setmainfont{Times New Roman} 

% Header Formatting
\pagestyle{fancy}
\fancyhf{}
\rhead{Your Last Name \thepage}
\lhead{}
\renewcommand{\headrulewidth}{0pt}

% MLA-style Title Formatting
\titleformat{\section}
  {\normalfont\large\bfseries}{\thesection}{1em}{}

% Paragraph Formatting
\setlength{\parindent}{0.5in}
\doublespacing

% Side Annotation Formatting
\newcommand{\annotate}[1]{\marginnote{\color{blue} #1}}

\begin{document}

% MLA Header
\noindent Your Name \\
Professor Name \\
Course Title \\
Date \\

% Title
\begin{center}
    \textbf{Title of Your Document}
\end{center}

\noindent
\annotate{Opening sentence should grab attention.}
Start your introduction here. When writing an MLA-style document, be sure to use a clear thesis and well-structured arguments. 

\annotate{Use citations properly to support claims.}
When citing sources, use proper MLA citation format (Author, page). For example, "This is an example quote" (Smith 23). 

\section*{Body Section 1}
\annotate{This paragraph introduces the main argument.}
Your first body paragraph should introduce the main argument and support it with evidence. 

\section*{Body Section 2}
\annotate{Consider counterarguments and address them here.}
A strong research paper considers counterarguments and addresses them thoughtfully. 

\section*{Conclusion}
\annotate{Restate the thesis and summarize key points.}
Your conclusion should reinforce the thesis and summarize key findings. Avoid introducing new ideas here.

\end{document}